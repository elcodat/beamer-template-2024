\documentclass[aspectratio=1610]{beamer}

\usepackage{Helmholtz-AI}

% References
\usepackage[
    backend=biber,
    url=true
    ]{biblatex}
\addbibresource{references.bib}

% Standard TeX math font. See also
%   https://github.com/elcorto/docwatch/blob/a8ea3d386a0db685cab89dee11879a76cc5e10a0/examples/template.yml#L21-L27
%%\usepackage{mathtools}
\usepackage{unicode-math}
\usepackage[olddefault]{fontsetup}
\DeclareMathAlphabet{\mathcal}{OMS}{cmsy}{m}{n}

\title[Short Title]{Super Awesome Project Title}
\subtitle{Subtitle}
\author{Max Mustermann}
\date{YYYY-MM-DD}
\institute{Helmholtz Center ABC}

\begin{document}

\maketitle
\maketitle[aeronautics-space-transport]
\maketitle[earth-environment]
\maketitle[energy]
\maketitle[health]
\maketitle[information]
\maketitle[matter]
\maketitle[computing]
\maketitle[chip]
\maketitle[abstract]

\begin{frame}
    \frametitle{Usage}

    \begin{enumerate}
        \item Download all files from Github\\~
        \item Edit \texttt{slides.tex} with your favorite editor\\~
        \item Compile the slides by either:\\~
        \begin{enumerate}
            \item Typing \texttt{make} or \texttt{latexmk} in the directory of \texttt{slides.tex} or\\~
            \item Using a LaTeX IDE like TeXstudio\\~
        \end{enumerate}
    \end{enumerate}
\end{frame}

\begin{frame}
    \frametitle{Main Slide Title}
    \framesubtitle{Subtitle with more details}

    \begin{itemize}
        \item Standard bullet \textbf{point} can be created with the \texttt{itemize} \textit{environment}
        \item They can have multiple sub-point
        \begin{itemize}
            \item As can be seen here
            \begin{itemize}
                \item Or here
            \end{itemize}
            \item The ordering is unimportant
        \end{itemize}
    \end{itemize}
\end{frame}


\begin{frame}
\frametitle{Equations}

    \begin{equation*}
        f(x) = \sum_i wx_i^2 + \frac{\beta}{2}
    \end{equation*}
\end{frame}


\begin{frame}
    \frametitle{Columns and Figures}

    \begin{columns}
        \begin{column}{0.4\textwidth}
            \begin{enumerate}
                \item Columns allow you to have side-by-side content\\~
                \item Each column itself is its own mini-slide\\~
                \item Figures can be imported by path\\~
                \item Scaling can be done relative to text width, height or initial size
            \end{enumerate}
        \end{column}
        \begin{column}{0.4\textwidth}
            \centering
            \includegraphics[width=0.8\textwidth]{logos/kit.pdf}
            \source{Karlsruhe Institute of Technology}
        \end{column}
    \end{columns}
\end{frame}


\begin{frame}[fragile]
    \frametitle{Source Code}

\begin{minted}{python}
import numpy as np

def foo(a, b):
    """
    asd
    """
    return a + b + 1
\end{minted}

\end{frame}

\begin{frame}
    \frametitle{Citations}
    
    \begin{itemize}
        \item Using the usual \texttt{cite} command \cite{debus2023reporting}\\~
        
        \item Using \texttt{fullcite}:
    \end{itemize}
    
    \vspace{1em}
    \fullcite{debus2023reporting}
\end{frame}


% command for making the color boxes
\newcommand\crule[3][black]{\textcolor{#1}{\rule{#2}{#3}}}

\begin{frame}
    \frametitle{Colors}
    \framesubtitle{Basic Definitions}

    \vspace{-1.0em}
    \begin{table}
        \centering
        \small
        \begin{tabular}{cl}
            \toprule
            \textbf{Color} & \textbf{Name}\\\midrule
            \crule[hgfblue]{10pt}{10pt} & hgfblue \\
            \crule[hgflightblue]{10pt}{10pt} & hgflightblue \\
            \crule[hgfdarkblue]{10pt}{10pt} & hgfdarkblue \\
            \crule[hgfmint]{10pt}{10pt} & hgfmint \\
            \crule[hgfhighlight]{10pt}{10pt} & hgfhighlight \\
            \crule[hgfpale]{10pt}{10pt} & hgfpale \\
            \crule[hgfgreen]{10pt}{10pt} & hgfgreen \\
            \crule[hgfgray]{10pt}{10pt} & hgfgray \\
            \crule[hgfaerospace]{10pt}{10pt} & hgfaerospace (short: hgfast) \\
            \crule[hgfearthandenvironment]{10pt}{10pt} & hgfearthandenvironment (short: hgfee) \\
            \crule[hgfenergy]{10pt}{10pt} & hgfenergy \\
            \crule[hgfhealth]{10pt}{10pt} & hgfhealth \\
            \crule[hgfinformation]{10pt}{10pt} & hgfinformation (short: hgfinfo) \\
            \crule[hgfmatter]{10pt}{10pt} & hgfmatter \\\bottomrule
        \end{tabular}
    \end{table}
\end{frame}


\begin{frame}
    \frametitle{Colors}
    \framesubtitle{Shades}

    \emph{For each color there exist 10 lighter shades, exemplary for hgfblue}\\

    \begin{table}
        \centering
        \small
        \begin{tabular}{cl}\toprule
            \textbf{Color} & \textbf{Name}\\\midrule
            \crule[hgfblue10]{10pt}{10pt} & hgfblue10 \\
            \crule[hgfblue20]{10pt}{10pt} & hgfblue20 \\
            \crule[hgfblue30]{10pt}{10pt} & hgfblue30 \\
            \crule[hgfblue40]{10pt}{10pt} & hgfblue40 \\
            \crule[hgfblue50]{10pt}{10pt} & hgfblue50 \\
            \crule[hgfblue60]{10pt}{10pt} & hgfblue60 \\
            \crule[hgfblue70]{10pt}{10pt} & hgfblue70 \\
            \crule[hgfblue80]{10pt}{10pt} & hgfblue80 \\
            \crule[hgfblue90]{10pt}{10pt} & hgfblue90 \\
            \crule[hgfblue]{10pt}{10pt} & hgfblue \\\bottomrule
        \end{tabular}
    \end{table}
\end{frame}


\begin{frame}
    \frametitle{Blocks}

    \begin{block}{block}
        This is how a regular block looks like
    \end{block}
    \vspace{2em}
    \begin{exampleblock}{exampleblock}
        An example block is stilled differently.
    \end{exampleblock}
    \vspace{2em}
    \begin{alertblock}{alertblock}
        Alert blocks can draw attention to critical information
    \end{alertblock}
\end{frame}


\section{Sections look like this}


\begin{frame}
    \frametitle{URLs and Fonts}

    \begin{itemize}
        \item There are raw links with the full URL \url{https://www.google.com}
        \item You can add also links with names \href{https://www.google.com}{Google}\\~
        
		\item You might also want to write in \hermann{Hermann Bold} - Helmholtz's title font\footnote{With footnotes like this.}
    \end{itemize}
\end{frame}


\begin{frame}
    \frametitle{On-slide References and Collaborators}

    \reference{
        Foo et al, ``Bar and its theories'', 42, HAICON 24.
    }
    % optional argument regulates the size of each of the included logos
    \collaborators[0.5cm]{{logos/dlr.pdf, logos/fzj.pdf, logos/helmholtz-munich.pdf, logos/hereon.pdf, logos/hzdr.pdf, logos/kit.pdf}}
\end{frame}


% add full list of references
\begin{frame}[allowframebreaks]{References}
    \printbibliography
\end{frame}

\end{document}
